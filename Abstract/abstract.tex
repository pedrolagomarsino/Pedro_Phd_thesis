\begin{abstract}
\addchaptertocentry{\abstractname} % Add the abstract to the table of contents

Encoding of spatial information in hippocapal place cells is believed to contribute to spatial cognition during navigation.
Whether the processing of spatial information is exclusively limited to neuronal cells or it involves other cell types, e.g. glial cells, in the brain is currently unknown. 
In this thesis work, I developed an analysis pipeline to tackle this question using statistical methods and Information Theory approaches. 
I applied these analytical tools to two experimental data sets in which neuronal place cells in the hippocampus were imaged using two-photon microscopy, while selectively manipulating astrocytic calcium dynamics with pharmacogenetics during virtual navigation.
Using custom analytical methods, we observed that pharmacogenetic perturbation of astrocytic calcium dynamics, through clozapine-n-oxyde (CNO) injection, induced a significant increase in neuronal place field and response profile width compared to control conditions.
The distributions of neuronal place field and response profile center were also significantly different upon perturbation of astrocytic calcium dynamics compared to control conditions.
Moreover, we found contrasting effect of astrocytic calcium dynamics perturbation on neuronal content of spatial information in the two data sets.
In the first data set, we found that CNO injection resulted in a significant increase in the average information content in all neurons. 
In the second data set, we instead found that mutual information values were not significantly different upon CNO application compared to controls.
Although the presented results are still preliminary and more experiments and analyses are needed, these findings suggest that astrocytic calcium dynamics may actively control the way hippocampal neuronal networks encode spatial information during virtual navigation. 
These data thus suggest a complex and tight interplay between neuronal and astrocytic networks during higher cognitive functions.

% \bigskip
% \noindent This latex project is a doctoral thesis template for the University of Hong Kong. The style and design of the entire project closely follow the official guidelines from the Graduate School: \href{https://intraweb.hku.hk/reserved_1/gradsch/PreparingandSubmittingYourThesis.pdf}{\textbf{Preparing and Submitting Your Thesis --- A Guide for MPhil and PhD Students.}} Generally, there is no strict stipulations on the style or format of different components of the thesis, except for the \textbf{Abstract}. According to the detailed regulations [\href{https://intraweb.hku.hk/reserved_1/gradsch/regulations_procedures/format_binding_presentation.pdf}{\textbf{Link}}], the \textbf{Abstract} should be part of the thesis with \uline{no fewer than 200 and no more than 500 words}. The format shall be the same as that of the thesis itself. The front page of each abstract shall contain the statement which includes:
% \begin{itemize}
%     \item Abstract of thesis entitled ``\dotuline{\hspace{8cm}}''
%     \item Submitted by \dotuline{\hspace{10cm}}~
%     \item for the degree of \dotuline{\hspace{9.5cm}}~
%     \item at the \univname~in (\usdate\today).
% \end{itemize}

% In addition to the opening of abstract, the abstract \uline{should appear before the title page}. The abstract in this template is \uline{not numbered, or counted in the pagination of the front matter, or listed in the table of contents}. All the requirements are fulfilled in this template.


% \bigskip
% \noindent \textbf{\Large How to adjust the typeset of Abstract}

% \noindent The typeset of the opening of abstract page is defined in the class file \codestyle{HKUThesis.cls} \textbf{Line 507-529}. Users can adjust the typeset by changing the settings. The layout of the main text is consistent with other parts of the thesis.


% \bigskip

% \noindent \textbf{Note that:} Considering the university may change its standards over time, users are not supposed to 100 percent ``trust'' this template. Even though the template is prepared strictly follow the stipulations of the Graduate School of The University of Hong Kong, this is \textbf{not an official} template and we are \textbf{not responsible} for any problems of your thesis submission caused by the format, style, typeset, etc of the template. We \textbf{strongly suggest} the users to read the latest \href{https://www.gradsch.hku.hk/gradsch/current-students/thesis-submission/guidelines-on-thesis-submission}{\textbf{Guidelines on Thesis Submission}} carefully and adjust the template accordingly to satisfy the stipulations of the university.

\end{abstract}